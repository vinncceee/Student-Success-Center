Discuss the state-of-the-art with respect to your product. What solutions currently exist, and in what form (academic research, enthusiast prototype, commercially available, etc.)? Include references and citations as necessary using the \textit{\textbackslash cite} command, like this \cite{Rubin2012}. If there are existing solutions, why won't they work for your customer (too expensive, not fast enough, not reliable enough, etc.). This section should occupy 1/2 - 1 full page, and should include at least 5 references to related work. All references should be added to the \textit{.bib} file, fully documented in IEEE format, and should appear in the \textit{references} section at the end of this document (the IEEE citation style will automatically be applied if your reference is properly added to the \textit{.bib} file).

ProTip: Consider using a citation manager such as Mendeley, Zotero, or EndNote to generate your \textit{.bib} file and maintain documentation references throughout the life cycle of the project.

%%%%%%%%%%%%%%%%%%%%%%%%
% To cite something, use the \cite command with the name of the bibtex entry in the curly braces.
% It will determine which reference number it is and insert that number where the \cite command is.
% e.g. \cite{Rubin2012}